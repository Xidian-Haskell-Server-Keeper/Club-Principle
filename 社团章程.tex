\documentclass[UTF8]{ctexart}

\usepackage{color,titlesec,hyperref}

\title{学生社团章程}
\titleformat{\section}[runin]{\raggedright\bfseries}{第\chinese{section} 章}{1em}{}
\titleformat{\subsection}[runin]{\raggedright\bfseries}{第\chinese{subsection} 条}{1em}{}
\titleformat{\subsubsection}[runin]{\raggedright\bfseries}{\qquad\quad(\chinese{subsubsection} )}{1em}{} 





\begin{document}
\maketitle
\newpage
\section{总则}
\subsection{社团名称}
西安电子科技大学Hackage镜像站维护社团,并称可西电Haskell-Server维护社区
\subsection{社团宗旨}
坚决遵守宪法及相关法律和法规、校团委的规定,遵守社会道德风尚。为校内师生提供Hackage镜像服务 \footnote{镜像来自 
\href{http://hackage.haskell.org}{hackage.haskell.org}}
及其相关的Hoogle
\footnote{Hoogle与 Hayoo 均为用于提供Haskell的公共Hackage中各类API使用方法的搜索服务。同Hackage一样,三者均在GitHub上开源。
\href{http://hoogle.haskell.org}{Hoogle的网站}
}
与Hayoo
\footnote{\href{http://hayoo.fh-wedel.de/}{Hayoo的网站}}
搜索服务。推动校内Haskell
\footnote{\href{http://www.haskell.org}{Haskell} 是一门函数式语言}
的发展与普及,促进函数式编程在校内的普及。
\subsection{社团职能}
\subsubsection{}建立并维护Hackage镜像站,向校内师生提供标准\verb"Hackage 2"
\footnote{即现在公共Hackage的版本。}
服务;
\subsubsection{}建立并维护Hoogle与Hayoo搜索站,向校内师生提供搜索服务。
\subsubsection{}向校内师生提供Haskell的基本入门指导、GHC的安装帮助、cabal的安装使用帮助、创建Hackage包的入门指导等帮助;
\subsubsection{}定期举行相关活动。

\section{组织机构与管理制度}
\subsection{}组织形式为全体监督下的核心会负责制,并以“社区”为社团组织形式。
\subsection{}社区负责者 \footnote{相当于社长} 一人,社区代表一人,各组设组长一名。
\subsection{}核心会由社团技术骨干与各类负责人组成。
\subsection{}社区负责人由社团全体从社团技术骨干\footnote{为主要人选。}与各类负责人中选举产生。
\subsection{}核心会的主要职能包括:
\subsubsection{}总结并向全体汇报这一时期的的社区各项总结
\footnote{具体由一名核心会成员负责。}
;
\subsubsection{}组织与社团相关的活动;


\end{document} 